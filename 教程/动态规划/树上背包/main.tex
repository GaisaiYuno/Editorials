\documentclass{article}
\usepackage{graphicx}
\usepackage{amsfonts}
\usepackage[utf8]{inputenc}
\usepackage{xeCJK}
\usepackage{amsmath}
\usepackage{amsthm}
\usepackage{bm}
\usepackage{minted}
\usepackage{hyperref}
\usepackage[dvipsnames]{xcolor}
\usepackage{parskip}
\usepackage{soul}
\usepackage{amssymb}
\usepackage{cancel}
\usepackage{ragged2e}
\usepackage[normalem]{ulem}
\usepackage[stable]{footmisc}
\newtheorem*{lemma*}{Lemma}


\title{树上01背包\footnote{更多内容请访问:\url{https://github.com/SamZhangQingChuan/Editorials}}}
\author{张晴川\\\href{mailto:qzha536@aucklanduni.ac.nz}{\texttt{qzha536@aucklanduni.ac.nz}}}

\begin{document}
\maketitle

\section{问题}
给一棵 $N$ 个点的有根树,每个节点 $i$ 有一个物品,体积为 $C_i$,价值为 $W_i$。如果选第 $i$ 个节点的物品,必须同时选他的父亲。给定背包容量上限 $V$,求最大价值。

复杂度要求 $O(NV)$

\section{定义}

\paragraph{泛化物品}
一个泛化物品 $f$ 是一个定义域为 $\{0,1\ldots V\}$,值域为实数的函数,也可以看成是长度为 $V+1$ 的数组。

\paragraph{泛化物品的并}
给定两个泛化物品 $f$ 和 $g$,定义泛化物品的并 $(f+g)$ 为 $(f+g)[i] = \max(f[i],g[i])$。

\paragraph{泛化物品的组合}
给定两个泛化物品 $f$ 和 $g$,定义泛化物品的组合 $(f*g)$ 为 $(f*g)[i] = \max_{0\le j \le i}(f[j]+g[i-j])$。

\section{一些特殊泛化物品}


\paragraph{乘法单位元 \textbf{1}}
定义 \textbf{1} 为 $[0,-\infty,\ldots,-\infty ]$,即在 $0$ 以外都取 $-\infty$。


\paragraph{指示函数 $I_{C,W}$}
定义指示函数 $I_{C,W}$ 为 $[-\infty,\ldots,W ,\ldots,-\infty ]$,即只 $C$处取 $W$, 其余取 $-\infty$。表示一个体积为 $C$,价值为 $W$ 的物品。


\section{性质}

\paragraph{乘法单位元}
对于任意物品 $f$,$f * \textbf{1} = f$。

证明:由定义显然。
\paragraph{交换律}
对于任意物品 $f,g$,$f+g = g+f$,$f*g = g*f$。

证明:由定义显然。
\paragraph{结合律}
对于任意物品 $f,g,h$,$(f+g)+h = f+(g+h)$,$(f*g)*h = f*(g*h)$。

证明:由定义显然。

\paragraph{分配律}
对于任意物品 $f,g,h$,$(f+g)*h = f*h + g*h$。

感性理解:先从 $f,g$ 里面选优的再与 $h$ 组合等价于先分别求 $f,g$ 与 $h$ 的组合再挑优的。

证明:

\begin{align*}
    ((f+g)*h)[i] 
    &= \max_{j = 0}^i ((f+g)[j] + h[i-j])\\
    &= \max_{j = 0}^i (\max(f[j],g[j]) + h[i-j])\\
    &= \max_{j = 0}^i (\max(f[j],g[j]) + h[i-j])\\
    &= \max_{j = 0}^i (\max(f[j]+h[i-j],g[j]+h[i-j]))\\
    &= \max(\max_{j = 0}^i(f[j]+h[i-j]),\max_{j = 0}^i(g[j]+h[i-j]))\\
    &= \max((f*h)[i],(g*h)[i])\\
    &= ((f*h)+(g*h))[i]\\
\end{align*}

\section{算法}

设以 $r$ 为根的子树对应的泛化物品为 $dp_r$,即 $dp_r[C]$ 等于这棵子树中总体积恰好为 $C$ 时的最大价值。这里我们规定必须要拿根 $r$对应的物品,即 $dp_r[i<C_r] = -\infty$。假设 $r$ 的儿子们分别为 $\{s_1,s_2,\ldots \}$。

那么不难发现如下关系:

$$
dp_r = I(C_r,W_r) * (\textbf{1}+dp_{s_1})* (\textbf{1}+dp_{s_2}) * \cdots
$$

由于计算组合每次是 $O(V^2)$ 的,总复杂度为 $O(NV^2)$,无法通过。

现在考虑如何增量加上一个儿子的贡献:$f' = f * (\textbf{1}+dp_s)$,其中 $f$ 的初值为 $I(C_r,W_r)$,即只选根节点。

展开可以得到如下结果:
\begin{center}
    $f' = f * (\textbf{1}+dp_s) = f*\textbf{1} + f*dp_s = f + f*dp_s$
\end{center}

$f$ 是已知量,所以只需要求出 $f*dp_s$。

利用乘法的结合律,我们可以换一个形式:
\begin{align*}
    f*dp_s
    &= f * (I(C_s,W_s) * (\textbf{1}+dp_{s_{1}^{\prime}})* (\textbf{1}+dp_{s_2^{\prime}}) * \cdots)\\
    &= (f * I(C_s,W_s)) * (\textbf{1}+dp_{s_{1}^{\prime}})* (\textbf{1}+dp_{s_2^{\prime}}) * \cdots)
\end{align*}

这相当于把 $s$ 的初值从 $I(C_s,W_s)$ 替换为 $f * I(C_s,W_s)$,由于 $I(C_s,W_s)$ 形式简单,这相当于强行在$f$中加入一个物品。这一步是 $O(V)$ 的,于是每多一个节点,复杂度只会多 $O(V)$。因此总复杂度是 $O(NV)$。

在DFS计算完 $f*dp_s$ 之后,通过 $O(V)$ 的时间更新 $f' = f+f*dp_s$ 即可。

最终答案为 $dp_{root}$ 中的最大值。

\section{代码}
\href{https://gist.github.com/SamZhangQingChuan/7c15974f308a45d6fcb114d029b4eda8}{HDOJ 3593}
\end{document}

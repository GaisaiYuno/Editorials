\documentclass{article}
\usepackage{graphicx}
\usepackage[utf8]{inputenc}
\usepackage{xeCJK}
\usepackage{amsmath}
\usepackage{bm}
\usepackage{minted}
\usepackage{hyperref}
\usepackage[dvipsnames]{xcolor}
\usepackage{parskip}
\usepackage{soul}
\usepackage{cancel}
\usepackage{ragged2e}
\usepackage[normalem]{ulem}

\hypersetup{hidelinks}

\title{欧几里得算法及其扩展\footnote{更多内容请访问:\url{https://github.com/SamZhangQingChuan/Editorials}}}
\author{张晴川\\\href{mailto:qzha536@aucklanduni.ac.nz}{\texttt{qzha536@aucklanduni.ac.nz}}}

\begin{document}
\maketitle
\section{问题}
\textbf{二元一次不定方程}:给定非负整数 $a,b,c(1 \le a,b,c \le 10^{18})$,求一组 $ax+by = c$ 的整数解 $(x,y)$。

\section{定义}
\begin{itemize}
\item $a\mid b$ 表示 $a$ 整除 $b$,即存在 $k$ 满足 $b = ka$
\item $\gcd(a,b)$ 表示 $a,b$ 的最大公约数 (greatest common divisor)。值得一提的是,最大公约数的定义其实是被所有公约数整除的数,不过在只考虑非负整数的情况下,也就相当于最大的那个公约数了。
\end{itemize}

\section{欧几里得算法(辗转相除法)}

设 $g$ 是 $a$ 和 $b$ 的最大公约数,那么 $ax+by$ 一定也是 $g$ 的倍数(为什么?)。所以如果 $c$ 不是 $g$ 的倍数,直接返回无解。所以我们先考虑如何计算最大公约数 $\gcd(a,b)$:

当 $b = 0$ 的时候,$\gcd(a,b) = \gcd(a,0) = a$。

而$b \neq 0$的时候,我们用 $\gcd(a,b) = \gcd(a-b,b)$ 来减小问题规模。以下是证明,一共分为两步:
\begin{enumerate}
\item $a,b$ 的公约数一定是 $a-b,b$ 的公约数,不会遗漏:
\begin{align*}
d \mid a,d \mid b 
&\implies a = a'd,b = b'd \\
&\implies (a-b) = (a'-b')d,b = b'd \\
&\implies d \mid (a-b), d \mid b
\end{align*}
\item $a-b,b$ 的公约数一定是 $a,b$ 的公约数,不会新增:
\begin{align*}
d \mid (a-b),d \mid b 
&\implies (a-b) = c'd,b = b'd \\
&\implies a = (c'+b')d,b = b'd \\
&\implies d \mid (a-b), d \mid b
\end{align*}

\end{enumerate}

于是得到$\gcd(a,b) = \gcd(a-b,b)$。注意到 $a\bmod b$ 相当于 $a$ 减去若干次 $b$,所以$\gcd(a,b) = \gcd(a\bmod b, b)$

\subsection*{代码}

欧几里得算法求最大公约数:
\inputminted[linenos,autogobble]{cpp}{gcd.cpp}


\section{扩展欧几里得算法}


\subsection*{解的存在性}

现在假设 $c$ 是 $\gcd(a,b)$ 的倍数,那么问题转化为如何解 $ax+by = \gcd(a,b)$,然后给解乘上 $\displaystyle \frac{c}{\gcd(a,b)}$ 即可。

\subsection*{具体流程}

假设存在两个等式:

\begin{align*}
    ax_1 + by_1 &= c_1\\
    ax_2 + by_2 &= c_2\\
\end{align*}

可以得到: 
\begin{align*}
    a(x1-x_2) + b(y_1-y_2) &= c_1-c_2\\
\end{align*}

所以等式也可以像整数一样加减乘除。我们只需要以下两个方程开始,不断辗转相除直到等式右边是 $\gcd(a,b)$ 即可。

\begin{align*}
    1a + 0b &= a\\
    0a + 1b &= b\\
\end{align*}
    
具体参考以下代码。
    
    
    

\subsection*{代码}
\inputminted[linenos,autogobble]{cpp}{exgcd.cpp}

\section{应用:乘法逆元}

\paragraph{问题}
给定一对互素的正整数 $a$ 和 $m$,即 $\gcd(a,m) = 1$。求一个整数 $a^{-1}$ 满足 $aa^{-1}\equiv 0 \mod m$。

\paragraph{解法}
首先我们解不定方程 $ax+my = 1$。由于互素,一定存在合法解。现在等式左边等于 $1$,所以等式左边等于 $1$,所以在模 $m$ 意义下一定也等于 $1$。由于 $my$ 是 $m$ 的倍数,所以在模 $m$ 意义下等于 $0$。那么就可以得到 $ax \equiv 1 \mod m$。

让 $a^{-1} = x$ 就是所求的逆元了。

\end{document}
